This paper succeeded in developing an AI prediction model to determine if a patient belongs in a specific risk group regarding knee osteoarthritis. Given data was preprocessed by filling missing values, normalizing and feature selection. Different versions of the SVM classifier was evaluated. Resulting in the optimal version having proportionality balanced class weights, a non-linear kernel and utilizing 21 of the original 23 features of the dataset. The performance of the resulting model was underwhelming in terms of precision. The source of the low performance is not clear. However, it might be caused by; model tuning, overfitting, data preprocessing or model selection. 

For future work it would be interesting to look into aquiring more processing power to try out more versions of the model and different data preprocessing steps. This could include a larger range of values of the model's parameters, different scaling techniques and other techniques to fill out missing values in the data. Outlier detection could be used to identify potential noisy data. Additionally, other models than SVM should be investigated.              

\unfinished{Write Conclusion}

\unfinished{Other model}

\unfinished{conclude in regards to clinical setting}