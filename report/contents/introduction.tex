
This paper will visit the development, training and evaluation of an AI prediction model. The model is to be used in a clinical setting regarding knee osteoarthritis treatment. A successful and effective treatment is the insertion of a Total Knee Arthroplasty (TKA). The number of TKA insertions are high in Denmark and it is expected to increase. TKA insertions are not always satisfactory and 5-10\% will need revision within 10 years. The objective of the model is individualized patient selection to identify patient's risk group. This enables optimization before, under and after surgery, tailoring  individual treatment plans and thereby increase the chance of the TKA surviving\cite{problem-description}.           

The paper is done as a project in the course Computer Vision and Machine Learning at Aarhus University. The dataset is given together with an explanation of the features and description of the problem. The dataset is in advance split in training, validation and test sets. The training and validation observations are labeled. As the model is to determine whether a patient belongs in a risk group or not, it is a supervised binary classification problem.

The model that will be used to solve the classification problem is a Support Vector Machine(SVM). Several preprocessing techniques will be utilized. As the data is incomplete some filling of missing values is needed. The data is imbalanced, therefore some measures are investigated to mitigate this. During model tuning cross-validation in combination with a grid search of parameters is utilized to select the optimal parameters.  
